\documentclass[pdflatex, a4paper,12pt]{article}
%\documentclass[a4paper,10pt]{scrartcl}

\usepackage[utf8x]{inputenc}

\title{Implementation of a Novel Beaconing System for Eduroam
 Inter-Campus Monitoring}
\author{Chris Orchard}
\date{12/10/2011}

\pdfinfo{%
  /Title    ()
  /Author   ()
  /Creator  ()
  /Producer ()
  /Subject  ()
  /Keywords ()

}
\setcounter{page}{44}

\begin{document}
\begin{center}
{\LARGE Development of Network Monitoring Tools for The Eduroam 
Wireless Roaming Service}\\[1em]

Author: Chris Orchard\\
Supervisor: Dr. Tim Chown
\end{center}

\section{Problem}


Eduroam is an academic wireless network deployed by many universities in the UK
and around the world, supported at an international level by
TERENA\footnote[1]{Trans-European Research and Education Networking Association:
www.terena.org}. The philosophy behind eduroam
is that any member of an academic
institution offering the eduroam wireless network will have internet access in
exactly the same manner in any location offering eduroam. In the UK, this
service is provided by JANET, the national interconnect for
universities, who maintain the national proxies used for inter-site eduroam
authentication. The support for monitoring of this inter-site eduroam functionality by system administrators or network
users is currently limited, with neither groups having access to detailed
availability and diagnostic data that could assist in the maintenance and
further adoption of eduroam in the UK.

\section{Goals}

The goal of this project is to create a two-part monitoring system, with a
server side RADIUS\footnote[2]{Remote Authentication Dial In User Service,
Defined in RFC
2865} beacon to allow administrators to diagnose issues with their
deployment and provide JANET with an overview of eduroam adoption progress, and
a physical client-mode monitor that JANET can deploy at eduroam sites to test
end-to-end connectivity via a local access-point. Basic visualisation utilities will
also be provided as part of the project. To support the design and implementation
stages, existing similar solutions will first be identified and evaluated to
gain the necessary domain knowledge. The \emph{dbeacon} multicast beacon has been identified
as a prime example for a server-side beacon.
 
\section{Scope}

There are several activities outside of the goal definition that are worthy of
consideration, due to their relevance to the project. Whilst these extra
features are out of scope and will not be implemented, the monitoring tools will
be designed to facilitate the future construction and integration of these items.
 The first is a
mobile device based user-orientated connectivity checker, where a user can
retrieve and contribute data about the service provision in their location. The
second is the utilisation of RadSec\footnote[3]{Draft standard:
http://tools.ietf.org/html/draft-ietf-radext-radsec-09}, an extension to the RADIUS specification
that adds improved inter-site functionality via TLS tunnelling over TCP. RadSec
is a possible successor to the existing national proxy system operated by JANET.

\end{document}
