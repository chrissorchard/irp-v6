\section{Introduction}

%An introduction with some history
In 1998 the IETF published the specification for the successor to the IPv4
networking protocol; IPv6. IPv6 was developed to overcome the shortcomings
of the IPv4 protocol, primarily the limit of around 3.7\e{9} total usable
addresses\cite{beijnum_ip} but also improves support for extensions, flow
control and for secure and private communication\cite{rfc2460}.

%initial hopes
Around the time the specification for IPv6 was published, it was believed that
the standard would be quickly adopted, and by 2001 IPv6 was supported by most
network hardware vendors and was being integrated into the operating systems of
the time\cite{huang_ipv6_2000}\cite{durand_deploying_2001}. It was accepted even
earlier than this that
the transition from IPv4 to IPv6 would not be simple however, and to this effect
the NGTrans working group produced guidance on deploying IPv6, releasing
documents on tunnelling\cite{rfc1933}\cite{rfc2893} and routing\cite{rfc2185}. 

%2008: oh balls
By 2008, 10 years after the publication of the original IPv6
specification, it had become clear that this initial hope of a fast
transition towards IPv6 had not been the case. Using measurements of IPv6 routing
data and web server access data, Huston finds that although IPv6 was being
deployed at a steady pace across the infrastructure of the internet, usage of
IPv6 by end users was very low, with IPv6 traffic observed being equivalent to
0.3\% of the amount of IPv4 traffic seen\cite{huston_ipv6_????}. In a similar vein, Domingues 
et al. conclude that despite the improvements IPv6 offered over IPv4
and the progress made by network hardware vendors and academic deployments,
global IPv6 deployment was not on track and that ISPs were shying away from
deploying IPv6 infrastructure\cite{domingues_is_2007}.

%2013: progress has begun once again.
Moving towards the present day the pace of IPv6 deployment has changed again, a
change that can be heavily attributed to the exhaustion of the IPv4 address
space at the start of 2011 when the Internet Assigned Numbers Authority (IANA)
allocated the last of its IPv4 addresses to the Regional Internet Registries.
Things: Allocation data from RIPE, Geoff Huston articles, world IPv6 day(s),
possible mention of PI space.

%TODO: talk about measuring deployment and why this is important
%TODO: link this back to earlier statements about progress of deployments
Collecting and storing data about the structure of the internet and how it is
used is essential to gauge the progress of IPv6 deployment, and mindful of this
fact many organisations and individuals have been collecting such data from an
early stage of the life of IPv6... Things: how this means we can attempt to draw
conclusions and follow deployment as described above. However, many of these
data sets and measurement techniques have drawbacks and limitations, something
that is also discussed extensively in academic analysis of the deployment of
IPv6. This paper attempts to draw together some of the approaches possible for
collecting data about the deployment of IPv6, how this data can and has been
analysed and what the advantages and difficulties of each approach might be,
and explore whether these measurements are sufficient and useful for progressing
the deployment of IPv6 and if the measurements can be used in combination to
draw further conclusions or guidance.

Things: Give a quick overview of the types of measurements being discussed.


%TODO: introduce what will be discussed, etc
NOTES:
Since then IPv6 deployment has increased as several large web companies have
deployed IPV6 with high profile events such as World IPv6
Day and World IPv6 Launch\footnote[2]{http://www.worldipv6launch.org/}.



