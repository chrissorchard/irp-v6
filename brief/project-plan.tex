\documentclass[pdflatex, a4paper,12pt]{article}
%\documentclass[a4paper,10pt]{scrartcl}

\usepackage[utf8x]{inputenc}

\begin{document}
\begin{center}
{\LARGE Deployment of IPv6 in Europe: Measurements, Motivations, and Challenges}\\[1em]

Author: Chris Orchard\\
Supervisor: Dr. Tim Chown
\end{center}

\paragraph{}

In 1998 the IETF published the specification for the successor to the IPv4
protocol; IPv6, developed to overcome the shortcomings of the IPv4 networking
protocol. The uptake of the IPv6 however did not see
widespread deployment until the primary shortcoming of IPv4, the limit of 3.7E9
total usable addresses [cite bgpexpert.com], came to the fore at the start of
2011 when the Internet Assigned Numbers Authority (IANA) allocated the last of
its IP addresses to the Regional Internet Registries. 

\paragraph{}

The IPv6 address space and the allocation of subnets of IPv6 address space
within Europe is administered by Réseaux IP Européens Network Coordination
Centre (RIPE NCC). 




\end{document}
