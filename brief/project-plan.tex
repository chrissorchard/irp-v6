\documentclass[pdflatex, a4paper,12pt]{article}
%\documentclass[a4paper,10pt]{scrartcl}

\usepackage[utf8x]{inputenc}

\providecommand{\e}[1]{\ensuremath{\times 10^{#1}}}

\begin{document}
\begin{center}
{\LARGE Deployment of IPv6 in Europe: Measurements, Motivations, and Challenges}\\[1em]

Author: Chris Orchard\\
Supervisor: Dr. Tim Chown
\end{center}

\paragraph{}

In 1998 the IETF published the specification for the successor to the IPv4
networking
protocol; IPv6. IPv6 was developed to overcome the shortcomings of the IPv4
protocol, primarily the limit of 3.7\e{9} total usable
addresses\footnote[1]{http://www.bgpexpert.com/addressespercountry.php},
an issue which came to a peak at the start of 2011 when the Internet Assigned Numbers Authority (IANA) 
allocated the last of its IPv4 addresses to the Regional Internet Registries.
Since then IPv6 deployment has increased as several large web companies have
deployed IPV6 with high profile events such as World IPv6
Day and World IPv6 Launch\footnote[2]{http://www.worldipv6launch.org/}.

\paragraph{}

The IPv6 address space and the allocation of subnets of IPv6 address space
within Europe is administered by Réseaux IP Européens Network Coordination
Centre (RIPE NCC). In addition to the registry service, RIPE NCC also provide a
range of tools and data sources to monitor and analyse internet infrastructure.

\paragraph{}

The primary goal of this project is to summarise existing research into the measurement
of deployment of IPv6, motivation behind the deployment of IPv6, and challenges
faced by organisations wanting to deploy IPv6, with a focus on the European area
administered by RIPE NCC. Secondary goals of the project are investigating the
use of active monitoring devices in collecting internet statistics,
investigating the effects of Provider Independent (PI) addressing on the
deployment of IPv6, and suggesting ways to improve the deployment of IPv6 within
Europe.





\end{document}
