\section{Conclusions}

To return once again to the current state of IPv6 deployment, the measurements
presented have shown that although progress of IPv6 deployment was poor until
2008, since then the rate of adoption has substantially improved. As interest in
the state of IPv6 deployment has grown, so has the number of efforts to track
its progress, with significant input from internet administrative organisations
and research organisations alike, and the addition of service availability and
usage data with historical records kept shows progress in measuring IPv6
deployment, with a view of deployment that is a much closer 
approximation of the intended goal of measuring transition from IPv4 to IPv6.

It is possible to conclude that the current state of the measurement of IPv6
deployment is good enough to give an accurate depiction of the how far the
internet has progressed with deployment of IPv6, and when combined the data
sources available correlate well and enable further analysis to be performed.

Another conclusion that can be made is that the presentation of data is
important. Access to data from RIPE NCC for prefix allocation and routing
advertisements is made easier by the substantial toolset provided for analysing
and producing graphs from the data, leading to a wealth of analysis on the data
being performed by researchers. The reason data presentation is important is
to help provide motivation to organisations to make the transition to supporting
IPv6. After 2008 when the imminent exhaustion of the IPv4 address space was
widely publicised and the RIRs and other internet organisations increased
attempts to motivate IPv6 deployment, both through adjustment of policies to
make the transition process easier and creating greater awareness of IPv6 and
its benefits, uptake of IPv6 increase significantly. It is reasonable to
conclude that further efforts to educate and increase awareness could further
improve the state of IPv6 deployment.

There are many possibilities for further work in measuring the deployment of
IPv6, as there is a lack of work done on combining data sets for further
analysis, and in particular combining the routing and allocation data with
service availability and usage data. Visibility and presentation of data is
important when trying to promote the deployment of IPv6 and further work to
create more informative displays from service availability data would improve
its effectiveness, particularly in regards to a strategy of naming and shaming
those countries that are lagging behind in the deployment of IPv6, as data shows
that those countries with national policies promoting the transition to IPv6
have much better uptake\cite{olivier_mj_crepin-leblond_ipv6_2011}. There could
also be some potential for
attempting to make a more formal combination of measurements as an accepted
metric for IPv6 deployment.

