\section{Introduction}

%An introduction with some history
In 1998 the IETF published the specification for the successor to the IPv4
networking protocol; IPv6. IPv6 was developed to overcome the shortcomings
of the IPv4 protocol, primarily the limit of around 3.7\e{9} total usable
addresses\cite{beijnum_ip_2013} but also improves support for extensions, flow
control and for secure and private communication\cite{rfc2460}.

%initial hopes
Around the time the specification for IPv6 was published, it was believed that
the standard would be quickly adopted, and by 2001 IPv6 was supported by most
network hardware vendors and was being integrated into the operating systems of
the time\cite{huang_ipv6_2000}\cite{durand_deploying_2001}. It was accepted even
earlier than this that
the transition from IPv4 to IPv6 would not be simple however, and to this effect
the NGTrans working group produced guidance on deploying IPv6, releasing
documents on tunnelling\cite{rfc1933}\cite{rfc2893} and routing\cite{rfc2185}. 

%2008: oh balls
By 2008, 10 years after the publication of the original IPv6
specification, it had become clear that this initial hope of a fast
transition towards IPv6 had not been the case. Using measurements of IPv6 routing
data and web server access data, Huston finds that although IPv6 was being
deployed at a steady pace across the infrastructure of the internet, usage of
IPv6 by end users was very low, with IPv6 traffic observed being equivalent to
0.3\% of the amount of IPv4 traffic seen\cite{huston_ipv6_2008}. In a similar vein, Domingues 
et al. conclude that despite the improvements IPv6 offered over IPv4
and the progress made by network hardware vendors and academic deployments,
global IPv6 deployment was ``not on track'' and that ISPs were shying away from
deploying IPv6 infrastructure\cite{domingues_is_2007}.

%2013: progress has begun once again.
Moving towards the present day the pace of IPv6 deployment has now increased,
something that can be attributed somewhat to the exhaustion of the IPv4 address
space, and the actions of the Regional Internet Registries (RIRs) to promote IPv6 more heavily in the run up to
this event. Since start of 2011 when the Internet Assigned Numbers Authority (IANA)
allocated the last of its IPv4 addresses to the Regional Internet
Registries\cite{number_resource_organisation_free_2011},
there have been several other efforts to improve awareness and promote
deployment of IPv6, including the ``World IPv6 Day'' and ``World IPv6 Launch''
events organised by the Internet Society (ISOC) where participants including a
number of large internet companies such as Google, Facebook and Cisco enabled
dual-stack access on their public facing services\footnote[1]{http://www.worldipv6launch.org/}.

Whilst it is clear that IPv6 is required for the future operation of the
internet\cite{huston_ipv6_2008}, the progress of the transition from IPv4  has not always proceeded
at the rate those who designed and implemented the protocol would wish. As the
organisations that maintain the infrastructure of the internet and those that
provide services over that infrastructure move from IPv4 to IPv6, it is clear
that accurately measuring and recording the status of the deployment of IPv6
is of benefit as a mechanism for trying to work out what went wrong with
the initial deployment efforts, motivating organisations that have not yet
transitioned to IPv6 to do so, and to try and predict what the future may hold
for the deployment of IPv6 and whether sufficient deployment can be reached before
the IPv4 network becomes unmaintainable\cite{huston_primer_2013}.

Many organisations and individuals have been collecting data to try and measure
the deployment of IPv6 from an
early stage in the life of IPv6, utilising a variety of sources and a range of
different metrics. This paper attempts to summarise those measurements and
the analysis that has been performed upon them. Some measurments investigated
are: allocation of IPv6 addresses from the RIRs, routes advertised to IPv6
networks on the internet,  
usage and availability data collected by users of the internet, and measurements
collected by distributed network probes.
The qualities and limitations of each method are assessed, and then an attempt
is made to correlate data from each of the methods together to attempt to verify
or draw further conclusions on the deployment of IPv6. 
%Methods of analysing the
%measurement data that have been demonstrated on older data are 
%also re-applied to data collected since the publication of the original
%analysis work. 
Finally, it is explored whether these
measurements are sufficient and useful for progressing
the deployment of IPv6 and how the measurements could be used in combination to
predict future developments or provide guidance to the community.

