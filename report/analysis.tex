\section{Analysis}

\subsection{Comparing Measurements}

The methods of collecting data to track IPv6 deployment presented all have
advantages and disadvantages, and the different approaches provide a view of
different facets of the deployment and use of IPv6 on the internet. 

Prefix allocation data gives a complete view of all the addresses allocated by
IANA and the RIRs, with all RIRs publishing the details of which prefixes they
have allocated and to whom they have allocated them to. Unfortunately there is
no way to gain a greater depth than RIR though, so no data is available on how
LIRs such as ISPs are allocating prefixes out their own prefixes. The allocation
data is made easily accessible by all the RIRs, and in particular RIPE NCC make
an effort to make their allocation data accessible and to provide tools to allow
analysis and generation of statistics on the data\cite{ripe_ncc_ripestat_2011}. Historical data for
past allocations is complete by definition as the RIRs must keep complete
records of all the prefix allocations they make to administer allocation. One of
the most important properties of a measurement approach is how well it
represents the state of IPv6 deployment, an area in which prefix allocation data
falls short, as the request for and allocation of prefixes to organisations does
not imply use of that prefix, and Karpilovsky et al. find that a significant
proportion of prefix allocated are never announced via BGP so are not
reachable\cite{karpilovsky_quantifying_2009}.
However, the data still reveals underlying trends in the deployment of IPv6 as
the proportion of LIRs that request address space and then advertise a route to
that prefix remains fairly static.

Routing data detailing the BGP routes announced by ASs across the internet
provides a good view of how IPv6 is supported on the backbone of the internet,
and the total amount of address space being advertised, but lacks data from the
edge of the internet. The BGP tables also vary depending on where the data is
collected from, as different nodes on the internet have differing view of the
structure of the internet, particularly if route aggregation is taken into
account. A number of projects make BGP routing data freely available, collecting
route information from a number of sources to get a more complete data set. As
BGP routing data is a well established source of data for performing
measurements of internet infrastructure, good historical data is generally
available, supported by both RIRs and IXPs. As a representation of the state of
deployment of IPv6, routing data gives a more accurate account of which IPv6
prefixed are in use, although route aggregation may limit the usefulness of data
in some cases, and even announcing a BGP route for an IPv6 prefix does not
imply that an LIR or any of its customers are providing any services from those
addresses.

Data for the availability of services and usage of those services over IPv6 is
much less complete, in part caused by the de-centralised nature of the internet
meaning that it is very difficult to capture data from all of the content
providers and consumers of the internet. However, it is possible to attain a
reasonably representative sample by looking at the more popular services
available, as the majority of traffic goes to or from these services making it
possible for companies like Google to collect usage data from a very wide user
base, and for service availability data to be collected by iterating through
publicly available lists of the most visited sites on the web. Service
availability data is also more difficult to collect than prefix allocation and
routing data, as it is necessary to actively probe service providers to find if
their services are available over IPv6. Despite the negative aspects of service
availability and usage data, it is a very good indicator of progress of IPv6
transition as the provision and use of services over IPv6 is the definition of
success for IPv6 deployment. Unfortunately most of the efforts to collect this
type of data are relatively recent in comparison to collection of routing and
allocation data, so historical records do not cover as far back as those for
routing and allocation data.

It is also possible to combine some approaches to permit further analysis,
particularly in tracking the how changes in IPv6 deployment progress from
organisations requesting new addresses, to advertising those addresses, to
providing services using those addresses, to users accessing those
services over IPv6.

\subsection{Analysis of IPv6 Deployment}

\subsubsection{Analysing Service Availability Data}

\subsubsection{Combining Measurements}

\subsubsection{Factors Affecting IPv6 Deployment}


%Returning to the discussion of the history of deployment of IPv6, it is clear
%that between 1998 when the specification of IPv6 was released and 2008 that the
%transition from IPv4 to IPv6 was much slower than the community had hoped(cite).
%Using the measurements and analysis thereof collected it is possible to propose
%some reasons why this may have been the case. 


%\subsection{Using Measurements to Promote Deployment}

%\item can we use different methods together to get a better picture of IPv6 deployment.
%\item recent changes, the current state of IPv6, what the measurements are currently telling us.
%\item can we in any way make any useful predictions about the future of IPv6 deployment (and has anyone else made reasonable forecasts previously)
%\item spend some time explaining why IPv6 is not currently as deployed as it could be.
