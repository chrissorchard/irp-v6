%
% irp-report.tex - main report file
%
% C. Orchard - 16/4/2013
%

% use draft mode for more spacing etc.
%\documentclass[draft, technote, a4paper]{IEEEtran}
% can increase to 11pt for cheeky page number bump
\documentclass[10pt, final, conference, a4paper]{IEEEtran}

%some packages, mostly from malucrawl report
\usepackage[pdftex]{graphicx}
\usepackage[utf8x]{inputenc}
\usepackage{cite}
\usepackage{hyperref}
\usepackage{listings}
\usepackage{nomencl}

\providecommand{\e}[1]{\ensuremath{\times 10^{#1}}}

\begin{document}

\title{Measuring the Deployment of IPv6}

\author{Chris Orchard,
\thanks{Chris Orchard is an undergraduate in the department of Electronics and
Computer Science at the University of Southampton} 
\thanks{Supervised by Dr. Tim Chown}
}


\maketitle

%TODO: Get Permission to use all the graphs from elsewhere.

\begin{abstract}
IPv6, the successor to the IPv4 networking protocol is now 15 years old, yet is
still not substantially deployed across the internet. This report explores the
approaches to measuring this deployment and the analysis performed upon those
measurements. Each technique is evaluated and compared, and the use of different
data sets in combination is investigated. It is found that the standard and
number of measurements of the deployment of IPv6 has improved since 2008 and
that this improvement is helping to drive further deployment of IPv6. 
\end{abstract}

\begin{IEEEkeywords}
IPv6, deployment, Provider-Independent address space, measurement
\end{IEEEkeywords}

%start of the document proper

\section{Introduction}

%An introduction with some history
In 1998 the IETF published the specification for the successor to the IPv4
networking protocol; IPv6. IPv6 was developed to overcome the shortcomings
of the IPv4 protocol, primarily the limit of around 3.7\e{9} total usable
addresses\cite{beijnum_ip_2013} but also improves support for extensions, flow
control and for secure and private communication\cite{rfc2460}.

%initial hopes
Around the time the specification for IPv6 was published, it was believed that
the standard would be quickly adopted, and by 2001 IPv6 was supported by most
network hardware vendors and was being integrated into the operating systems of
the time\cite{huang_ipv6_2000}\cite{durand_deploying_2001}. It was accepted even
earlier than this that
the transition from IPv4 to IPv6 would not be simple however, and to this effect
the NGTrans working group produced guidance on deploying IPv6, releasing
documents on tunnelling\cite{rfc1933}\cite{rfc2893} and routing\cite{rfc2185}. 

%2008: oh balls
By 2008, 10 years after the publication of the original IPv6
specification, it had become clear that this initial hope of a fast
transition towards IPv6 had not been the case. Using measurements of IPv6 routing
data and web server access data, Huston finds that although IPv6 was being
deployed at a steady pace across the infrastructure of the internet, usage of
IPv6 by end users was very low, with IPv6 traffic observed being equivalent to
0.3\% of the amount of IPv4 traffic seen\cite{huston_ipv6_2008}. In a similar vein, Domingues 
et al. conclude that despite the improvements IPv6 offered over IPv4
and the progress made by network hardware vendors and academic deployments,
global IPv6 deployment was not on track and that ISPs were shying away from
deploying IPv6 infrastructure\cite{domingues_is_2007}.

%2013: progress has begun once again.
Moving towards the present day the pace of IPv6 deployment has now increased,
something that can be attributed somewhat to the exhaustion of the IPv4 address
space, and the actions of the RIRs to promote IPv6 more heavily in the run up to
this event. Since start of 2011 when the Internet Assigned Numbers Authority (IANA)
allocated the last of its IPv4 addresses to the Regional Internet Registries,
there have been several other efforts to improve awareness and promote
deployment of IPv6, include two "World IPv6 Day" events where a collection of
large internet companies switched to dual-stack configurations on their primary
services\footnote[1]{http://www.worldipv6launch.org/}.

Whilst it is clear that IPv6 is required for the future operation of the
internet\cite{huston_ipv6_2008}, the progress of the transition from IPv4  has not always proceeded
at the rate those who designed and implemented the protocol would wish. As the
organisations that maintain the infrastructure of the internet and those that
provide services over that infrastructure move from IPv4 to IPv6, it is clear
that accurately measuring and recording the status of the deployment of IPv6
is of benefit as a mechanism for trying to work out what went wrong with
the initial deployment efforts, motivating organisations that have not yet
transitioned to IPv6 to do so, and to try and predict what the future may hold
for the deployment of IPv6, whether sufficient deployment can be reached before
the IPv4 network becomes unmaintainable\cite{huston_primer_2013}.

Many organisations and individuals have been collecting such data from an
early stage of the life of IPv6 utilising a variety of sources and a range of
different metrics, and this paper attempts to summarise those measurements and
the analysis that has been performed upon them. The specific methods investigated
are: prefix allocation data from the RIRs, BGP data from internet routers, and
usage and availability data collected by users of the internet.
The qualities and limitations of each method are assessed, and then an attempt
is made to correlate data from each of the methods together to attempt to verify
or draw further conclusions on the deployment of IPv6. Analysis that has
been performed on older data is also re-applied to new data collected since the
analysis was originally performed. Finally, it is explored whether these
measurements are sufficient and useful for progressing
the deployment of IPv6 and how the measurements could be used in combination to
predict future developments or provide guidance to the community.


\section{Data Sources for IPv6 Deployment Measurements}

\begin{itemize}
    \item BGP Route Data
    \begin{itemize}
        \item sources
        \item analysis of data
        \begin{itemize}
            \item What has been done?
            \item How can we do this?
        \end{itemize}
        \item difficulties and potential problems with the date
    \end{itemize}
    \item Prefix Allocation
    \begin{itemize}
        \item same specifics as BGP Routes
        \item PI vs. PA space, can we measure this?
        \item can relate allocation -> use?
    \end{itemize}
    \item Usage Data
    \begin{itemize}
        \item same specifics as BGP Routes
        \item can we measure the extent of tunneling?
        \item happy eyeballs
        \item browser configuration (re: dual stack)
    \end{itemize}
    \item Active Measurements
    \begin{itemize}
        \item same specifics as BGP Routes
        \item Atlas and Ark
        \item how does this add to the existing other data sets?
    \end{itemize}
\end{itemize}

\section{Analysis}

\subsection{Comparing Measurements}

The methods of collecting data to track IPv6 deployment presented all have
advantages and disadvantages, and the different approaches provide a view of
different facets of the deployment and use of IPv6 on the internet. 

Prefix allocation data gives a complete view of all the addresses allocated by
IANA and the RIRs, with all RIRs publishing the details of which prefixes they
have allocated and to whom they have allocated them to. Unfortunately there is
no way to gain a greater depth than RIR though, so no data is available on how
LIRs such as ISPs are allocating prefixes out their own prefixes. The allocation
data is made easily accessible by all the RIRs, and in particular RIPE NCC make
an effort to make their allocation data accessible and to provide tools to allow
analysis and generation of statistics on the data\cite{ripe_ncc_ripestat_2011}. Historical data for
past allocations is complete by definition as the RIRs must keep complete
records of all the prefix allocations they make to administer allocation. One of
the most important properties of a measurement approach is how well it
represents the state of IPv6 deployment, an area in which prefix allocation data
falls short, as the request for and allocation of prefixes to organisations does
not imply use of that prefix, and Karpilovsky et al. find that a significant
proportion of prefix allocated are never announced via BGP so are not
reachable\cite{karpilovsky_quantifying_2009}.
However, the data still reveals underlying trends in the deployment of IPv6 as
the proportion of LIRs that request address space and then advertise a route to
that prefix remains fairly static.

Routing data detailing the BGP routes announced by ASs across the internet
provides a good view of how IPv6 is supported on the backbone of the internet,
and the total amount of address space being advertised, but lacks data from the
edge of the internet. The BGP tables also vary depending on where the data is
collected from, as different nodes on the internet have differing view of the
structure of the internet, particularly if route aggregation is taken into
account. A number of projects make BGP routing data freely available, collecting
route information from a number of sources to get a more complete data set. As
BGP routing data is a well established source of data for performing
measurements of internet infrastructure, good historical data is generally
available, supported by both RIRs and IXPs. As a representation of the state of
deployment of IPv6, routing data gives a more accurate account of which IPv6
prefixed are in use, although route aggregation may limit the usefulness of data
in some cases, and even announcing a BGP route for an IPv6 prefix does not
imply that an LIR or any of its customers are providing any services from those
addresses.

Data for the availability of services and usage of those services over IPv6 is
much less complete, in part caused by the de-centralised nature of the internet
meaning that it is very difficult to capture data from all of the content
providers and consumers of the internet. However, it is possible to attain a
reasonably representative sample by looking at the more popular services
available, as the majority of traffic goes to or from these services making it
possible for companies like Google to collect usage data from a very wide user
base, and for service availability data to be collected by iterating through
publicly available lists of the most visited sites on the web. Service
availability data is also more difficult to collect than prefix allocation and
routing data, as it is necessary to actively probe service providers to find if
their services are available over IPv6. Despite the negative aspects of service
availability and usage data, it is a very good indicator of progress of IPv6
transition as the provision and use of services over IPv6 is the definition of
success for IPv6 deployment. Unfortunately most of the efforts to collect this
type of data are relatively recent in comparison to collection of routing and
allocation data, so historical records do not cover as far back as those for
routing and allocation data.

It is also possible to combine some approaches to permit further analysis,
particularly in tracking the how changes in IPv6 deployment progress from
organisations requesting new addresses, to advertising those addresses, to
providing services using those addresses, to users accessing those
services over IPv6.

\subsection{Analysis of IPv6 Deployment}

\subsubsection{Analysing Service Availability Data}

An aspect of service availability data that has not been widely investigated is
the trends in the data over time; Prior does not keep historical data, and no
analysis has yet been performed on the service availability data set collected
by Crépin-Leblond at \verb+ipv6matrix.org+. Data for the IPv6 Matrix is
collected and stored separately for each top level domain (TLD) and each data set
is archived by time of measurement completion. The analysis presented here is
based upon that presented by Crépin-Leblond in
\cite{olivier_mj_crepin-leblond_ipv6_2011}, finding the percentage of hosts
surveyed that support access to web, mail, DNS or NTP using IPv6, shown in
Figure (ref) for the .com, .uk, .net, .org, and .de TLDs. The graph shows
similar results to the other data sources, with a sharp increase in number of
IPv6 enabled services in mid-2011. The increase shown in Figure (ref) can almost
certainly be attributed to the success of ``World IPv6 Launch'' as the largest
single increase occurs in the first scan completed after the 6th of June 2012
for each TLD, correlating with the number of content providers that enabled IPv6
for their services permanently on this date.

\subsubsection{Factors Affecting IPv6 Deployment}

As discussed in the introduction, to ease the challenges faced by organisations
wanting to move away from IPV4 to IPv6, the NGtrans working group was formed by
the IETF to produce recommendations for transition tools and policies. Many
proposals were generated, with some proposals becoming standards, but little
guidance given on how to use the transition mechanisms effectively. The 6to4
tunnelling mechanism designed to allow hosts to connect to IPv6 services over
IPv4 specified in 2001 (cite), is a good example of a transition mechanism that
has quite possibly hindered deployment of IPv6 rather than helping. 6to4 relies
on a network of anycast relays to interface between the IPv4 and IPv6 internet,
a feature intended to allow automatic tunnel configuration, but which in
practice causes the protocol to be very unstable as not all relays can reach the
entire IPv6 internet and operation of the relays is not regulated. Aben finds
that around 15\% of connections made using 6to4 fail (cite), an unacceptable
level for content providers considering switching their services to dual-stack.
Figure \ref{fig:google-access} shows that 6to4 usage has become negligible,
mitigating this issue however 

Steffann et al.
identify 16 different mechanisms for tunnelling IPv6 traffic over IPv4

Transistion Mechanisms

ISP Motivation

PI vs PA

Performance

Reliability

FUD

\subsubsection{Combining Measurements}
%Returning to the discussion of the history of deployment of IPv6, it is clear
%that between 1998 when the specification of IPv6 was released and 2008 that the
%transition from IPv4 to IPv6 was much slower than the community had hoped(cite).
%Using the measurements and analysis thereof collected it is possible to propose
%some reasons why this may have been the case. 


%\subsection{Using Measurements to Promote Deployment}

%\item can we use different methods together to get a better picture of IPv6 deployment.
%\item recent changes, the current state of IPv6, what the measurements are currently telling us.
%\item can we in any way make any useful predictions about the future of IPv6 deployment (and has anyone else made reasonable forecasts previously)
%\item spend some time explaining why IPv6 is not currently as deployed as it could be.

\section{Conclusions}

To return once again to the current state of IPv6 deployment, the measurements
presented have shown that although progress of IPv6 deployment was poor until
2008, since then the rate of adoption has substantially improved. As interest in
the state of IPv6 deployment has grown, so has the number of efforts to track
its progress, with significant input from internet administrative organisations
and research organisations alike, and the addition of service availability and
usage data with historical records kept shows progress in measuring IPv6
deployment, with a view of deployment that is a much closer 
approximation of the intended goal of transition from IPv4 to IPv6.

It is possible to conclude that the current state of the measurement of IPv6
deployment is good enough to give an accurate depiction of the how far the
internet has progressed with deployment of IPv6, and when combined the data
sources available correlate well and enable further analysis to be performed.

Another conclusion that can be made is that the presentation of data is
important. Access data from RIPE NCC for prefix allocation and routing
advertisements is made easy by the substantial toolset provided for analysing
and producing graphs from the data, leading to a wealth of analysis on the data
being performed by researchers. The reason data presentation is important is
to help provide motivation to organisations to make the transition to supporting
IPv6. After 2008 when the imminent exhaustion of the IPv4 address space was
widely publicised and the RIRs and other internet organisations increased
attempts to motivate IPv6 deployment, both through adjustment of policies to
make the transition process easier and creating greater awareness of IPv6 and
its benefits, uptake of IPv6 increase significantly. It is reasonable to
conclude that further efforts to educate and increase awareness could further
improve the state of IPv6 deployment.

There are many possibilities for further work in measuring the deployment of
IPv6, as there is a lack of work done on combining data sets for further
analysis, and in particular combining the routing and allocation data with
service availability and usage data. Visibility and presentation of data is
important when trying to promote the deployment of IPv6 and further work to
create more informative displays from service availability data would improve
its effectiveness, particularly in regards to a strategy of naming and shaming
those countries that are lagging behind in the deployment of IPv6, as data shows
that those countries with national policies promoting the transition to IPv6
have much better uptake\cite{olivier_mj_crepin-leblond_ipv6_2011}. There could also be some potential for
attempting to make a more formal combination of measurements as an accepted
metric for IPv6 deployment.



\section{Acknowledgement}

The author would like to thank Dr Tim Chown for his support and supervision
throughout this project, and to Olivier MJ Crépin-Leblond and Christian de
Larrinaga from ISOC England for their input and granting full access to the
data at \verb+www.ipv6matrix.org+.

%bibliography

\bibliographystyle{IEEEtran}
%\bibliography{irp-report,rfc,i-d}{}
\bibliography{irp-report-zotero,rfc,i-d}{}

\end{document}
