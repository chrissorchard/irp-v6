\section{Introduction}

In 1998 the IETF published the specification for the successor to the IPv4
networking protocol; IPv6. IPv6 was developed to overcome the shortcomings
of the IPv4 protocol, primarily the limit of 3.7\e{9} total usable
addresses\cite{beijnum_ip} but also improves support for extensions, flow
control and for secure and private communication\cite{rfc2460}.

Despite the advantages offered by IPv6 over IPv4 deployment during the
intervening years has been limited and the pace of
uptake of IPv6 by end users has been disappointingly slow,  
%possibly graph here?, also numbers, fact and quote filling
... suggesting that ..., and ... also .... More
recently, IPv6 deployment has started moving more quickly, widely put down to
the exhaustion of the IPv4 address space at the start of 2011 when the Internet
Assigned Numbers Authority (IANA) allocated the last of its IPv4 addresses to
the Regional Internet Registries[cite, quote]. As of [date] IPv6 traffic still only makes up
about [some-amount] of the total traffic on the internet, with the majority of
the rest still IPv4[cite].

%TODO: discuss motivation, talk about organisations involved.

%TODO: talk about measuring deployment and why this is important
%TODO: link this back to earlier statements about progress of deployments

%TODO: introduce what will be discussed, etc

Since then IPv6 deployment has increased as several large web companies have
deployed IPV6 with high profile events such as World IPv6
Day and World IPv6 Launch\footnote[2]{http://www.worldipv6launch.org/}.


\begin{itemize}
    \item What is IPv6?
    \item Why should we use it? (also IPv4 exhaustion)
    \item Overview of the current state of IPv6 deployment.
    \item Ask questions
    \item Enumerate types of measurement, sources of data
    \item Citation\cite{rfc3177}
\end{itemize}
