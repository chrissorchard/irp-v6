\section{Conclusions}


To return once again to the current state of IPv6 deployment, substantial
progress has been made in the last 5 years, with large organisations such as
Google and Facebook having enough confidence in the reliability, performance,
and future availability of IPv6 to enable dual-stack networking on their
services, and the DNS root servers also deploying a dual-stack solution. The
allocation of IPv6 prefixes, advertisement of routes in the BGP routing tables,
availability of services and use of those services all continue to increase
steadily. The movement away from IPv4 to IPv6 only solutions has not been
explored in this report, but given the current progress of IPv6 deployment it
would be reasonable to conclude that it will be some time before IPv6 will reach
sufficient deployment for ISPs and content providers to stop considering IPv4 as
the primary networking protocol and instead primarily deploy IPv6 natively
without native IPv4 using transition mechanisms such as NAT64 to communicate
with IPv4 users.

As interest in
the state of IPv6 deployment has grown, so has the number of efforts to track
its progress, with significant input from Internet administrative organisations
and research organisations alike, and the addition of service availability and
usage data with historical records kept shows progress in measuring IPv6
deployment, with a view of deployment that is a much closer 
approximation of the intended goal of measuring transition from IPv4 to IPv6.
It is possible to conclude that the current state of the measurement of IPv6
deployment is good enough to give an accurate depiction of the how far the
Internet has progressed with deployment of IPv6, and when combined the data
sources available correlate well and enable further analysis to be performed.

Whilst the measurements methods being used are of a high quality, some
recommendations can be made to improve how the data can be used and made
available. Presentation of the measurement data is a field in which some
organisations invest much more effort than others. Access to data from RIPE
NCC for prefix allocation and routing
advertisements is improved by the substantial toolset provided for analysing
and producing graphs from the data, leading to a greater amount of analysis on the data
being performed by researchers. An approach such as this to allowing access to
analysis of the data across other data sources and collection methods may
provide motivation for further study on the deployment of IPv6, promoting
further deployment of IPv6. Whilst RIPE NCC make a substantial set of graphs
available, they do not make all of the data used to create the graphs easily
available. Availability of data collected from measurements is an area where
substantial improvements could be made, as currently the only measurement method
with anything approaching a standard data format is the BGP routing data, which
is largely supplied in the MRT format, a binary format that requires a C library
to decode. A possible improvement that could be made to providing access to
prefix allocation data is to use an open linked data approach, as the nature of prefix
allocation data is well suited to open linked data, with a large amount of data
that is appended to but rarely modified. The accessibility of usage data is a
significant issue, with Google only making a single graph of their data publicly
available. Improved policies to share data in agreed formats would make the
compilation and analysis of data from multiple sources using different methods
substantially easier. 

 After 2008 when the imminent exhaustion of the IPv4 address space was
widely publicised and the RIRs and other Internet organisations increased
attempts to motivate IPv6 deployment, both through adjustment of policies to
make the transition process easier and creating greater awareness of IPv6 and
its benefits, uptake of IPv6 increase significantly. It is reasonable to
conclude that further efforts to educate and increase awareness could further
improve the state of IPv6 deployment.

There are many possibilities for further work in measuring the deployment of
IPv6, as there is a lack of work done on combining data sets for further
analysis, for example using the prefix allocation and BGP routing data to
investigate whether the recent increase in allocation of PI space is reflected by
advertisement of prefixes in the PI space. Another possible combined measurement
is to track IPv6 blocks from allocation though advertisement to service
availability, and investigating patterns in the time between these events.
Visibility and presentation of data is
important when trying to promote the deployment of IPv6 and further work to
create more informative displays from service availability data would improve
its effectiveness, particularly in regards to a strategy of naming and shaming
those countries that are lagging behind in the deployment of IPv6, as data shows
that those countries with national policies promoting the transition to IPv6
have much better uptake\cite{olivier_mj_crepin-leblond_ipv6_2011}. There could
also be some potential for
attempting to make a more formal combination of measurements as an accepted
metric for IPv6 deployment.



